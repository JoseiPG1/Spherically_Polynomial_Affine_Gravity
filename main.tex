\documentclass{article}
\usepackage[utf8]{inputenc}
\usepackage{amsmath}
\usepackage{amsfonts}
\usepackage{cite} 
\usepackage{breqn}
\usepackage{graphicx}
\usepackage{hyperref}

\allowdisplaybreaks

%Useful definitions
\providecommand{\Rie}[3]{\mathcal{R}_{#1}{}^{ #2}{}_{#3}}
\providecommand{\ctG}[3]{\Gamma_{#1}{}^{ #2}{}_{#3}}
\providecommand{\ctg}[3]{\gamma_{#1}{}^{ #2}{}_{#3}}
\providecommand{\B}[3]{\mathcal{B}_{#1}{}^{ #2}{}_{#3}}
\providecommand{\P}[3]{\mathcal{P}_{#1}{}^{ #2}{}_{#3}}
\providecommand{\A}[1]{\mathcal{A}_{#1}}
\providecommand{\Ri}[1]{\mathcal{R}_{#1}}
\providecommand{\deV}[1]{\mathrm{d}V^{#1}}


\title{Spherically solutions of Polynomial Affine Gravity in the torsion-free sector}
\author{Jos\'e Perdiguero G\'arate}

\begin{document}

\maketitle
%\tableofcontents

\section{Introduction}
\label{sec:Introduction}



\section{Polynomial Affine Gravity}
\label{sec:PAG}


\subsection{Model review}
\label{subsec: model_review}

The Polynomial Affine Gravity uses the affine connection to mediated 
gravitational interactions, instead of the metric tensor. To build 
the action, we decompose the affine connection into its irreducible
fields
\begin{equation}
\begin{aligned}
    \label{affine_connection}
    \hat{\Gamma}_{\alpha}{}^{\beta}{}_{\gamma} & = \hat{\Gamma}_{(\alpha}{}^{\beta}{}_{\gamma)} +  \hat{\Gamma}_{[\alpha}{}^{\beta}{}_{\gamma]},  \\
    & = \ctG{\alpha}{\beta}{\gamma} + \B{\alpha}{\beta}{\gamma} + \delta^{\beta}_{[\gamma}\A{\alpha]},
\end{aligned}
\end{equation}
where the first term $\Gamma$ stands for the symmetric part and 
$\mathcal{A}$, $\mathcal{B}$ fields are related to the skew symmetric
part of the affine connection. 

Additionally, is necessary to introduce a volume form without
the use of the metric tensor. In order to achieve this, we used
the volume element defined as $\mathrm{d}V^{\alpha \beta \gamma \delta} 
= \mathrm{d}x^\alpha \wedge \mathrm{d}x^\beta \wedge \mathrm{d}x^\gamma
\wedge \mathrm{d}x^\delta$, which is completely antisymmetric.

Moreover, we want to preserve the invariance under diffeomorphism, 
which is why the symmetric part of the affine connection, goes 
indirectly through the covariant derivative $\nabla^\Gamma$. 

The action is built up using a sort of \textit{dimensional analysis}
technique. This strategy allows us to generate every scalar density
composed by powers of the irreducible fields of the affine connection.
This method has been using to build the action in four dimensions, see
Refs. and in three dimensions Ref [].

The most general action (up to topological invariants and boundary terms) 
in four dimensions is given by
\begin{equation}
    \label{PAG_action}
    \begin{split}
    S
    & =
    \int  \mathrm{d}V^{\alpha \beta \gamma \delta} \bigg[
    B_1 \mathcal{R}_{\mu\nu}{}^{\mu}{}_{\rho}\mathcal{B}_{\alpha}{}^{\nu}{}_{\beta}\mathcal{B}_{\gamma}{}^{\rho}{}_{\delta}
    + B_2 \mathcal{R}_{\alpha\beta}{}^{\mu}{}_{\rho} \mathcal{B}_{\gamma}{}^{\nu}{}_{\delta} \mathcal{B}_{\mu}{}^{\rho}{}_{\nu}
    + B_3 \mathcal{R}_{\mu\nu}{}^{\mu}{}_{\alpha} \mathcal{B}_{\beta}{}^{\nu}{}_{\gamma} \mathcal{A}_\delta
    \\
    & \quad
    + B_4 \mathcal{R}_{\alpha\beta}{}^{\sigma}{}_{\rho}\mathcal{B}_{\gamma}{}^{\rho}{}_{\delta}\mathcal{A}_\sigma
    + B_5 \mathcal{R}_{\alpha \beta}{}^{\rho}{}_{\rho} \mathcal{B}_{\gamma}{}^{\sigma}{}_{\delta} \mathcal{A}_\sigma
    + C_1 \mathcal{R}_{\mu\alpha}{}^{\mu}{}_{\nu} \nabla_\beta \mathcal{B}_{\gamma}{}^{\nu}{}_{\delta}
    \\
    & \quad
    + C_2 \mathcal{R}_{\alpha\beta}{}^{\rho}{}_{\rho} \nabla_\sigma \mathcal{B}_{\gamma}{}^{\sigma}{}_{\delta}
    + D_1 \mathcal{B}_{\nu}{}^{\mu}{}_{\lambda} \mathcal{B}_{\mu}{}^{\nu}{}_{\alpha} \nabla_\beta \mathcal{R}_{\gamma}{}^{\lambda}{}_{\delta}
    + D_2 \mathcal{B}_{\alpha}{}^{\mu}{}_{\beta} \mathcal{B}_{\mu}{}^{\lambda}{}_{\nu} \nabla_{\lambda} \mathcal{B}_{\gamma}{}^{\nu}{}_{\delta}
    \\
    & \quad
    + D_3 \mathcal{B}_{\alpha}{}^{\mu}{}_{\nu}\mathcal{B}_{\beta}{}^{\lambda}{}_{\gamma} \nabla_\lambda \mathcal{B}_{\mu}{}^{\nu}{}_{\delta}
    + D_4 \mathcal{B}_{\alpha}{}^{\lambda}{}_{\beta}\mathcal{B}_{\gamma}{}^{\sigma}{}_{\delta}\nabla_\lambda \mathcal{A}_\sigma
    + D_5 \mathcal{B}_{\alpha}{}^{\lambda}{}_{\beta} \mathcal{A}_\sigma \nabla_\lambda \mathcal{B}_{\gamma}{}^{\sigma}{}_{\delta}
    \\
    &\quad
    + D_6 \mathcal{B}_{\alpha}{}^{\lambda}{}_{\beta}\mathcal{A}_\gamma \nabla_\lambda A_\delta
    + D_7\mathcal{B}_{\alpha}{}^{\lambda}{}_{\beta} \mathcal{A}_\lambda \nabla_\gamma A_\delta
    + E_1\nabla_\rho \mathcal{B}_{\alpha}{}^{\rho}{}_{\beta} \nabla_\sigma \mathcal{B}_{\gamma}{}^{\sigma}{}_{\delta}
    \\
    &\quad
    + E_2 \nabla_\rho \mathcal{B}_{\alpha}{}^{\rho}{}_{\beta} \nabla_\gamma \mathcal{A}_\delta
    + F_1 \mathcal{B}_{\alpha}{}^{\mu}{}_{\beta} \mathcal{B}_{\gamma}{}^{\sigma}{}_{\delta} \mathcal{B}_{\mu}{}^{\lambda}{}_{\rho} \mathcal{B}_{\sigma}{}^{\rho}{}_{\lambda}
    + F_2\mathcal{B}_{\alpha}{}^{\mu}{}_{\beta} \mathcal{B}_{\gamma}{}^{\nu}{}_{\lambda} \mathcal{B}_{\delta}{}^{\lambda}{}_{\rho} \mathcal{B}_{\mu}{}^{\rho}{}_{\nu}
    \\
    &\quad
    + F_3 \mathcal{B}_{\nu}{}^{\mu}{}_{\lambda} \mathcal{B}_{\mu}{}^{\nu}{}_{\alpha} \mathcal{B}_{\beta}{}^{\lambda}{}_{\gamma} \mathcal{A}_\delta
    + F_4 \mathcal{B}_{\alpha}{}^{\mu}{}_{\beta}\mathcal{B}_{\gamma}{}^{\nu}{}_{\delta}\mathcal{A}_\mu \mathcal{A}_\nu \bigg].
    \end{split}
\end{equation}
Notice the Riemann curvature and the Ricci tensor are defined with respect to the symmetric part
of the connection.

Although the action written in Eq. \eqref{PAG_action} is clearly more complicated than
the Einstein-Hilbert action, there are several features

\subsection{Building the ansatz}

To build the ansatz of the affine connection $\Gamma$ we compute its Lie derivative 
$\mathcal{L}_{\xi_j}\ctG{\alpha}{\beta}{\gamma}$ along the Killing vectors $\xi_j$ that 
generate the desired symmetry, in this case a spherical symmetry. This procedure
has been extensively cover in Refs. [] 

The final affine connection's coefficient in the torsion-free sector are given by
twelve independent functions defined as follow:
\begin{equation}
\begin{aligned}
    \ctG{t}{t}{t} & = V(t,r) & \ctG{t}{t}{r} & = A(t,r) & \ctG{r}{t}{r} & = W(t,r) & \ctG{\theta}{t}{\theta} & = X(t,r) & \ctG{\phi}{t}{\phi} & = X(t,r)\sin^2\theta \\
    \ctG{t}{r}{t} & = B(t,r) & \ctG{t}{r}{r} & = Y(t,r) & \ctG{r}{r}{r} & = C(t,r) & \ctG{\theta}{r}{\theta} & = F(t,r) & \ctG{\phi}{r}{\phi} & = F(t,r)\sin^2\theta 
\end{aligned}
\end{equation}
Notice that, under a parametrization of the time coordinate, the first coefficient
can be set equal to zero, see Ref []. The above ansatz can be simplified even further 
by imposing additional symmetries on the affine connection. First, lets consider 
stationary solutions
\begin{equation}
    \nabla_{e_t}e_t = e_t\ctG{t}{t}{t} + e_r\ctG{t}{r}{t}
\end{equation}
\begin{equation}
    \nabla_{-e_t}\left(-e_t\right) = -e_t\ctG{t}{t}{t} + e_r\ctG{t}{r}{t}
\end{equation}
the consistent condition requires that $\ctG{t}{t}{t} = 0$. Applying the same principle
to the other basis vectors, then
\begin{align}
    \ctG{t}{j}{i} & = 0 & \ctG{i}{t}{j} & = 0
\end{align}
where $i$, $j$ are restricted to space index.

Next, we demand an azimuthal angle symmetry, in which case the
\begin{align}
    \ctG{\phi}{\phi}{\phi} & = 0 & \ctG{t}{j}{\phi} & = 0 & \ctG{i}{\phi}{j} & = 0
\end{align}

\subsection{Field equations}



\section{Solutions}
\label{sec:solutions}


\section{Final remarks}
\label{sec:final_remarks}


\end{document}